\newcommand{\AUTHOR}{Timo Friman, Henri Meltaus}
\newcommand{\DATE}{15.-16.9.2011}

\documentclass[hyperref={pdfauthor=\AUTHOR},14pt]{beamer}

\usepackage[finnish]{babel}
\usepackage[utf8]{inputenc}
\usepackage[T1]{fontenc}
\PassOptionsToPackage{hyphens}{url}\usepackage{listings}
\usepackage[hyphens]{url}

\usecolortheme{crane}
\usetheme{default}

\definecolor{mygreen}{rgb}{0,0.4,0}
\definecolor{myid}{rgb}{0.1,0.1,0.1}
\lstdefinestyle{Java}{language=java,
 basicstyle=\small,%\ttfamily,
 numbers=left,stepnumber=1,numberstyle=\small\ttfamily,
 numbersep=5pt,frame=tlbr,extendedchars=true,
 commentstyle=\color{mygreen}\ttfamily,
 %% stringstyle=\color{red}\ttfamily,
 stringstyle=\color{magenta},
 keywordstyle=\color{violet}\bfseries,
 ndkeywordstyle=\color{yellow}\bfseries,
 identifierstyle=\color{myid},
 % sensitive=false,
 basicstyle=\scriptsize,
}

\author{\AUTHOR}
\newcommand{\TITLE}{Spring Framework AOP}
\title[\TITLE]{\TITLE}

\date{\DATE}

\lstset{ % set listing language
language=XML,
breaklines=true,
breakautoindent=false,
frame=single,
%basicstyle=\tiny,
basicstyle=\scriptsize,
showspaces=false,               % show spaces adding particular underscores
showstringspaces=false
}

\begin{document}
\begin{frame}[plain]
\titlepage
\end{frame}

\begin{frame}[t, fragile]{Mitä AOP tarkoittaa?}
\begin{itemize}
\item Crosscutting concern
\item ``OO handles 85 \%, AOP 15\%"
\item Pelkkä businesslogiikka
\item Käyttöesimerkkejä: turvaominaisuuksien pakottaminen, auditointi, seuranta, transaktiotuki, cachetus
\end{itemize}
\end{frame}

\begin{frame}[t, fragile]{AOP'n peruskäsitteet}
\begin{itemize}
\item Join point
\item Pointcut
\item Advice
\item Introduction, esittely
\item Weaving
  \begin{itemize}
    \item Source
    \item Compile
    \item Loadtime
    \item Proxy (Spring way)
  \end{itemize}
\end{itemize}
\end{frame}

\begin{frame}[t, fragile]{AOP Advice}
\begin{itemize}
\item Kertoo mitä ja milloin
\item Before
\item After
\item After-returning
\item After-throwing
\item Around
\item Edellämainitut ovat käytössä Spring AOP:ssa, lisäksi on muitakin (mitä?)
\end{itemize}
\end{frame}

\begin{frame}[t, fragile]{AOP Pointcut}
\begin{itemize}
\item Valitsin joinpointeille
\item Spring AOP tukee vain executea
\item AspectJ tukee muitakin; control flow... listaa tähän nuo pointcutit
\end{itemize}
\end{frame}

\begin{frame}[t, fragile]{AspectJ}
\begin{itemize}
\item ``oikea'' AOP-toteutus
\item Oma syntaksi, laajentaa Javaa, heikohko IDE-tuki
\item Oma kääntäjä
\item Laaja, mahdollistaa aika huikeita asioita
\end{itemize}
\end{frame}

\begin{frame}[t, fragile]{Spring AOP}
\begin{itemize}
\item Proxypohjainen (JDK tai CGLib)
\item Pointcuteista käytössä vain metodin suoritus (execute)
\item Spring AOP'ssa on  alijoukko AspectJ'stä, sama formaatti
\item Mikäli Spring AOP ei riitä, voi ottaa käyttön AspectJ'n ja käyttää Springiä esimerkiksi aspektien riippuvuuksien hoitamiseen
\end{itemize}
\end{frame}

\begin{frame}[t, fragile]{Spring AOP jatkuu}
\begin{itemize}
\item paketti org.springframework.aop
\item paketti org.springframework.aop.interceptor tarjoaa muutaman valmiin interceptorin, joilla voi logittaa ja traceta esim.
\item ProxyFactoryBean käytetään Springin sisäisesti paljon, sitä voi käyttää myös ulkoisesti
\end{itemize}
% \begin{lstlisting}
%    <bean id="bean1" class="com.houston.HelloBean" />
%    <bean name="beanA, beanB" class="com.houston.WorldBean"/>
% \end{lstlisting}

\end{frame}



\end{document}