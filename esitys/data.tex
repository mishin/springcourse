\newcommand{\AUTHOR}{Timo Friman, Henri Meltaus}
\newcommand{\DATE}{15.-16.9.2011}

\documentclass[hyperref={pdfauthor=\AUTHOR},14pt]{beamer}

\usepackage[finnish]{babel}
\usepackage[utf8]{inputenc}
\usepackage[T1]{fontenc}
\PassOptionsToPackage{hyphens}{url}\usepackage{listings}
\usepackage[hyphens]{url}

\usecolortheme{crane}
\usetheme{default}

\definecolor{mygreen}{rgb}{0,0.4,0}
\definecolor{myid}{rgb}{0.1,0.1,0.1}
\lstdefinestyle{Java}{language=java,
 basicstyle=\small,%\ttfamily,
% numbers=left,stepnumber=1,numberstyle=\small\ttfamily,
 numbersep=5pt,frame=tlbr,extendedchars=true,
 commentstyle=\color{mygreen}\ttfamily,
 %% stringstyle=\color{red}\ttfamily,
 stringstyle=\color{magenta},
 keywordstyle=\color{violet}\bfseries,
 ndkeywordstyle=\color{yellow}\bfseries,
 identifierstyle=\color{myid},
 % sensitive=false,
 basicstyle=\scriptsize,
}

\lstset{ % set listing language
language=XML,
breaklines=true,
breakautoindent=false,
breakatwhitespace=true,
frame=single,
%basicstyle=\tiny,
basicstyle=\scriptsize,
showspaces=false,               % show spaces adding particular underscores
showstringspaces=false,
resetmargins=true
}

\definecolor{links}{HTML}{2A1B81}
\hypersetup{colorlinks,linkcolor=,urlcolor=links}

\author{\AUTHOR}
\newcommand{\TITLE}{Spring Framework Data Access}
\title[\TITLE]{\TITLE}

\date{\DATE}

\begin{document}
\begin{frame}[plain]
\titlepage
\end{frame}

\begin{frame}[t, fragile]{Transaktiot}
\begin{itemize}
\item Sama ohjelmointimalli riippumatta käytetäänkö global (JTA) tai local transaktioita
\item Ohjelmallinen transaktionhallinta \\
-- TransactionTemplate \\
-- PlatformTransactionManager
\item Deklaratiivinen transaktionhallinta\\
-- Perustuu AOP:hen\\
-- Konfigurointi XML:ssä tai annotaatioilla
\end{itemize}
\end{frame}

\begin{frame}[t, fragile]{PlatformTransactionManager}
\begin{itemize}
\item Springin transaktiohallinnan perusta 

\lstset{language=Java,style=Java}
\begin{lstlisting}
TransactionStatus getTransaction(TransactionDefinition definition) throws TransactionException
void commit(TransactionStatus status) throws TransactionException
void rollback(TransactionStatus status) throws  TransactionException
\end{lstlisting}
\item Useita toteutuksia
\begin{itemize}
\item JtaTransactionManager-toteutukset eri sovelluspalvelimille
\item DataSourceTransactionManager JDBC DataSource
\end{itemize}
\end{itemize}
\end{frame}

\begin{frame}[t, fragile]{TransactionDefinition}
\begin{itemize}
\item Isolation -- Miten eristetty transaktio on muista transaktioista
\item  Propagation -- Luodaanko uusi transaktio vai käytetäänkö olemassa olevaa
\item Timeout -- Kauanko transaktio voi olla käynnissä ennen automaattista rollbackia
\item Read-only status -- Kun tarvitaan vain luku-operaatio
\end{itemize}
\end{frame}

\begin{frame}[t, fragile]{TransactionStatus}
\lstset{language=Java,style=Java}
\begin{lstlisting}    
    boolean isNewTransaction()
    boolean hasSavepoint()
    void setRollbackOnly()
    boolean isRollbackOnly()
    void flush()
    boolean isCompleted()
\end{lstlisting}

\end{frame}

\begin{frame}[t, fragile]{Esimerkkikonfiguraatio}

\begin{lstlisting}
<bean id="dataSource" class="org.apache.commons.dbcp.BasicDataSource" destroy-method="close">
  <property name="driverClassName" value="${jdbc.driverClassName}" />
  <property name="url" value="${jdbc.url}" />
  <property name="username" value="${jdbc.username}" />
  <property name="password" value="${jdbc.password}" />
</bean>

<bean id="txManager" class="org.springframework.jdbc.datasource.
DataSourceTransactionManager">
 <property name="dataSource" ref="dataSource"/>
</bean>
\end{lstlisting}
\end{frame}

\begin{frame}[t, fragile]{@Transactional}
\begin{itemize}
\item Käyttöönotto XML:ssä <tx:annotation-driven transaction-manager="txManager"/>
\item Metodeihin, joissa tarvitaan transaktionhallintaa
\item Parametrit
\begin{itemize}
\item value = Käytettävä TransactionManager
\item propagation
\item isolation
\item readOnly
\item timeout
\item rollbackFor, rollbackForClassname = Poikkeukset joiden on aiheutettava rollback
\item noRollbackFor, noRollbackForClassname =  Poikkeukset jotka eivät saa aiheutettaa rollbackia

\end{itemize}
\end{itemize}
\end{frame}

\begin{frame}[t, fragile]{DAO support}
\begin{itemize}
\item Yhtenäinen tapa toteuttaa teknologiariippumattomia DAOja
-- JDBC, Hibernate, JDO, JPA
\item Alkuperäiset poikkeukset wrapataan Springin omiin DataAccessException-poikkeuksiin
\item @Repository-annotaation käyttö DAO-luokissa

\end{itemize}
\end{frame}

\begin{frame}[t, fragile]{JDBC Support}
\begin{itemize}
\item Huolehtii boilerplate koodista
\begin{itemize}
\item Tietokantayhteyksien avaaminen ja sulkeminen
\item Statementien luominen ja suorittaminen
\end{itemize}
\item API:n käyttäjälle jää 
\begin{itemize}
\item Kyselyjen muodostaminen
\item Tulosten iteroiminen
\end{itemize}
\item Useita luokkia
\begin{itemize}
\item JdbcTemplate
\item NamedParameterJdbcTemplate
\item SimpleJdbcTemplate
\item SimpleJdbcInsert
\item SimpleJdbcCall
\end{itemize}
\item Käyttö esim. DAO-luokkien sisällä
\end{itemize}
\end{frame}

\begin{frame}[t, fragile]{Template-luokat}
\begin{itemize}
\item JdbcTemplate
\begin{itemize}
\item Kaikki operaatiot: select, insert, update, delete, batch...
\item Muut template-luokat käyttävät tätä taustalla
\end{itemize}
\item NamedParameterJdbcTemplate
\begin{itemize}
\item Tuki nimetyille parametreille SQL-kyselyissä
\item Esim. select name from person where id = :personId
\end{itemize}
\item SimpleJdbcTemplate
\begin{itemize}
\item Yksinkertaisempi API
\item Yleisimmät JdbcTemplaten ja NamedParameterJdbcTemplate operaatiot
\end{itemize}
\end{itemize}
\end{frame}

\begin{frame}[t, fragile]{Muut luokat}
\begin{itemize}
\item SimpleJdbcInsert ja SimpleJdbcCall
\begin{itemize}
\item Hyödyntävät tietokannasta saatavaa meta-tietoa
\item Vähemmän konfiguroitavaa
\end{itemize}
\item SqlQuery, MappingSqlQuery, SqlUpdate, StoredProcedure
\begin{itemize}
\item Mallintavat sql-operaatiot olioiksi 
\end{itemize}
\end{itemize}
\end{frame}

% \begin{frame}{}
% \begin{itemize}
% \item
% \end{itemize}
% \end{frame}

\end{document}