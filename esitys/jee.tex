\newcommand{\AUTHOR}{Timo Friman, Henri Meltaus}
\newcommand{\DATE}{15.-16.9.2011}

\documentclass[hyperref={pdfauthor=\AUTHOR},14pt]{beamer}

\usepackage[finnish]{babel}
\usepackage[utf8]{inputenc}
\usepackage[T1]{fontenc}
\PassOptionsToPackage{hyphens}{url}\usepackage{listings}
\usepackage[hyphens]{url}

\usecolortheme{crane}
\usetheme{default}

\definecolor{mygreen}{rgb}{0,0.4,0}
\definecolor{myid}{rgb}{0.1,0.1,0.1}
\lstdefinestyle{Java}{language=java,
 basicstyle=\small,%\ttfamily,
 numbers=left,stepnumber=1,numberstyle=\small\ttfamily,
 numbersep=5pt,frame=tlbr,extendedchars=true,
 commentstyle=\color{mygreen}\ttfamily,
 %% stringstyle=\color{red}\ttfamily,
 stringstyle=\color{magenta},
 keywordstyle=\color{violet}\bfseries,
 ndkeywordstyle=\color{yellow}\bfseries,
 identifierstyle=\color{myid},
 % sensitive=false,
 basicstyle=\scriptsize,
}

\lstset{ % set listing language
language=XML,
breaklines=true,
breakautoindent=false,
breakatwhitespace=true,
frame=single,
%basicstyle=\tiny,
basicstyle=\scriptsize,
showspaces=false,               % show spaces adding particular underscores
showstringspaces=false,
resetmargins=true
}

\definecolor{links}{HTML}{AA1B81}
\hypersetup{colorlinks,linkcolor=,urlcolor=links}

\author{\AUTHOR}
\newcommand{\TITLE}{Spring Framework JEE  jne}
\title[\TITLE]{\TITLE}

\date{\DATE}

\begin{document}
\begin{frame}[plain]
\titlepage
\end{frame}

\begin{frame}[t, fragile]{Namespaceista 1/2}
Namespacet helpottavat elämää, niiden käyttö ei tosin ole pakollista,
vaan asiat voisi langoittaa itsekin XML:ssä.
\begin{itemize}
\item context
\item jee
\item p, xmlns:p="http://www.springframework.org/schema/p"\\
\begin{lstlisting}
<bean id="foo" class="example.Bean" p:message="Message passing using p-namespace">
\end{lstlisting}
\end{itemize}
\end{frame}

\begin{frame}[t, fragile]{Namespaceista 2/2}
\begin{itemize}
\item aop
\item util
\item jms
\item omat laajennokset
\begin{itemize}
\item  META-INF/spring.handler, spring.schema. 
\item überjar-pakkausongelmat, Maven shade-plugin etc.
\item Apache Mina FTP Server esimerkkinä laajennoksesta
\end{itemize}
\end{itemize}
\end{frame}

\begin{frame}{Templatet}
Springissä on monille perusasioille Template, joka hoitaa likaiset työt ja höyrykattilan levytykset piilossa. Niitä kannattaa käyttää, koodi selkeytyy usein jopa luettavaksi ja yksikkötestattavuus nousee.

Tässä on listattuna useimmiten käytetyt templatet:
\begin{itemize}
\item Tietokanta
\item Rest
\item Jms
\item Sähköposti
\end{itemize}
\end{frame}

\begin{frame}{JMS}
\begin{itemize}
\item Piilottaa todella hyvin JMS'n monimutkaisuuden
\item \href[colorlinks=true]{http://static.springsource.org/spring/docs/3.0.x/javadoc-api/org/springframework/jms/core/JmsTemplate.html}{JmsTemplate}
  lähettämiseen ja synkroniseen vastaanottoon
\item DefaultMessageListenerContainer hoitaa MDB-osuuden (MD Pojo) eli
  asynkronisen vastaanoton. Katso
  \href[colorlinks=true]{http://static.springsource.org/spring/docs/3.0.x/javadoc-api/org/springframework/jms/listener/AbstractMessageListenerContainer.html}{Javadocista} käyttöohjeita.
\end{itemize}
\end{frame}

\begin{frame}{Sähköposti}
\begin{itemize}
\item Helppo SMTP-lähetys esimerkiksi käyttäen JNDI:n läpi MailSessionia.
\item tukee MIMEä 
\item\href[colorlinks=true]{http://static.springsource.org/spring/docs/3.0.x/javadoc-api/org/springframework/mail/javamail/JavaMailSender.html}{JavaMailSender}
\end{itemize}
\end{frame}

\begin{frame}[t, fragile]{JNDI}
\begin{itemize}
\item JEE-namespacessa
\item helppokäyttöinen
\begin{lstlisting}
<jee:jndi-lookup id="myEmf"
            jndi-name="persistence/myPersistenceUnit"/>

<jee:jndi-lookup id="simple"
             jndi-name="jdbc/MyDataSource"
             cache="true"
             resource-ref="true"
             lookup-on-startup="false"
             expected-type="com.myapp.DefaultFoo"
             proxy-interface="com.myapp.Foo"/>
\end{lstlisting}
\end{itemize}
\end{frame}

\begin{frame}[t, fragile]{JMX}
\begin{itemize}
\item Export
\begin{lstlisting}
<context:mbean-export server="myMBeanServer" default-domain="myDomain"/>
\end{lstlisting}
\item Tuki serverin käynnistämiselle ja ``automaattiselle''
  etsimiselle
\begin{lstlisting}
<bean id="jmx.server"  class=
"org.springframework.jmx.support.MBeanServerFactoryBean"
   p:locateExistingServerIfPossible="true"/>
\end{lstlisting}

\end{itemize}
\end{frame}

\begin{frame}[t, fragile]{EJB}
\begin{itemize}
\item Tämäkin on tuettu
\item jee-namespacessa stateless session beanien hakemiseen tagit
\begin{lstlisting}
<jee:remote-slsb id="complexRemoteEjb"
    jndi-name="ejb/MyRemoteBean"
    business-interface="com.foo.service.RentalService"
    cache-home="true"
    lookup-home-on-startup="true"
    resource-ref="true"
    home-interface="com.foo.service.RentalService"
    refresh-home-on-connect-failure="true"/>
\end{lstlisting}
\end{itemize}
\end{frame}

\begin{frame}{Security}
\begin{itemize}
\item Ex-Acegi
\item Pääasiallisesti web, controller-taso
\item myös metoditaso (AOP)
\item JDBC, LDAP, inMemory
\end{itemize}
\end{frame}

\begin{frame}{Remoting}
\begin{itemize}
\item Invokers
\begin{itemize}
\item RMI
\item Hessian (bin), Burlap (asc) (caucho.com)
\item HTTP (Java serialization, Spring only)
\item WS
\end{itemize}
\item Itse käyttäisin useimmiten SOAP-WS:ää ja Apache CXF:ää näiden
  sijasta.
\end{itemize}
\end{frame}

\begin{frame}{Ajastus}
\begin{itemize}
\item Mahdollistaa tehtävien ajastamisen springin kontekstista tai
  annotaatioilla
\item TaskExecutor ja TaskScheduler -rajapinnat abstrahoivat Java 5,
  6, EE:n erot pois
\item Tuki sekä Quartz'lle että JDK:n Timerille
\end{itemize}
\end{frame}


\end{document}