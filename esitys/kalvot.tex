\newcommand{\AUTHOR}{Timo Friman, Henri Meltaus}
\newcommand{\DATE}{15.-16.9.2011}

\documentclass[hyperref={pdfauthor=\AUTHOR},14pt]{beamer}

\usepackage[finnish]{babel}
\usepackage[utf8]{inputenc}
\usepackage[T1]{fontenc}
\PassOptionsToPackage{hyphens}{url}\usepackage{listings}
\usepackage[hyphens]{url}

\usecolortheme{crane}
\usetheme{default}

\definecolor{mygreen}{rgb}{0,0.4,0}
\definecolor{myid}{rgb}{0.1,0.1,0.1}
\lstdefinestyle{Java}{language=java,
 basicstyle=\small,%\ttfamily,
 numbers=left,stepnumber=1,numberstyle=\small\ttfamily,
 numbersep=5pt,frame=tlbr,extendedchars=true,
 commentstyle=\color{mygreen}\ttfamily,
 %% stringstyle=\color{red}\ttfamily,
 stringstyle=\color{magenta},
 keywordstyle=\color{violet}\bfseries,
 ndkeywordstyle=\color{yellow}\bfseries,
 identifierstyle=\color{myid},
 % sensitive=false,
 basicstyle=\scriptsize,
}

\lstset{ % set listing language
language=XML,
breaklines=true,
breakautoindent=false,
breakatwhitespace=true,
frame=single,
%basicstyle=\tiny,
basicstyle=\scriptsize,
showspaces=false,               % show spaces adding particular underscores
showstringspaces=false,
resetmargins=true
}

\definecolor{links}{HTML}{2A1B81}
\hypersetup{colorlinks,linkcolor=,urlcolor=links}

\author{\AUTHOR}
\newcommand{\TITLE}{Spring Framework}
\title[\TITLE]{\TITLE}

\date{\DATE}

\begin{document}
\begin{frame}[plain]
\titlepage
\end{frame}

\begin{frame}[t, fragile]{Sisällysluettelo}
\begin{itemize}
\item Springin tausta
\item Inversion of Control aka. Dependency Injection
\item Springin hallinnoimat resurssit vs. new-operaattorilla luodut
\end{itemize}
\end{frame}

\begin{frame}{Springin historia ja tausta}
\begin{itemize}
\item kehitys aloitettu 2002, ensimmäinen versio julkaistiin 2004
\item tavoitteena helpottaa J(2)EE:n kanssa toimimista ja vähentää
  höyrykattilan levytystä... esimerkkeinä JNDI ja EJB remoting/transaktiot
\item mahdollisimman kajoamaton, tosin esimerkiksi annotaatiot ja
  apuluokat esimerkiksi yksikkötesteissä ovat yleisiä
\item POJO, rajapintojen käyttö
\item Ei sidottu mihinkään sovelluspalvelimeen tai sovelluskehikoihin,
  tarkoituksena mahdollistaa parhaat käytännöt ja parhaiden
  sovellusten käyttö
\end{itemize}
\end{frame}

\begin{frame}{Riippuvuuksien puskeminen}
\begin{itemize}
\item \url{http://www.martinfowler.com/articles/injection.html}
% http://www.adam-bien.com/roller/abien/entry/what_is_the_relation_between
\item pull vs push
% tarkoittaa sitä, että pull-mallissa new foo(), FooFactory.getFoo()
% tai jndi:n kautta
% push tarkoittaa sitä, että jokin olion ulkopuolelta puskee siihen
% riippuvuudet. tuota puskemista kutsutaan dependency injectioniksi
%
% IoC vs DI... IoC on yleisempi käsite kuin DI? IoC tarkoittaa
% riippuvuuksien langoittamista, näin eri sovelluskerrokset saadaan
% yhteen
\item etuja: löyhä kytkentä \& korkea koheesio $ \Rightarrow $ parempi
  ymmärrettävyys, uudelleenkäytettävyys ja modulaarisuus samoin kuin testattavuus
% loose coupling suomeksi?
\item JSR-299, JSR-330, javax.inject
\item setter vs konstruktori vs suora injektio kenttään
\end{itemize}
\end{frame}

\begin{frame}{Spring laajennukset}
Springissä on wrapperit ja apuluokat monille tunnetuille
teknologioille ja sovelluskehyksille, tässä listattuna niistä muutama:
\begin{itemize}
\item JDBC, JMS, JMX
\item JPA; Hibernate, Toplink, JDO, iBatis
\item Ajastus: Quartz
\item Templatetyökalut: Freemarker, Velocity
\item Struts, Tapestry
\item OXM; JAXB, JavaBeans, Castor 
\item AOP: AspectJ
\item Web Services; oma ratkaisu
\item Security (LDAP integration)
\end{itemize}
\end{frame}

\begin{frame}{IDE-tuki}
\begin{itemize}
\item XML/Java, joten tuettu valtaosin
\item Eclipse (Spring IDE)
\item IntelliJ Idea, NetBeans
\end{itemize}
\end{frame}


\begin{frame}{Springin modulit 1/2}
Springin ytimen lisäksi on olemassa useita aliprojekteja:
\begin{itemize}
\item Spring Security
\item Spring Web Flow
\item Spring Web Services (contract first)
\item Spring Integration 
\item Spring Batch
\item Spring Social
\item Spring Mobile (Spring Android)
\end{itemize}
\end{frame}

\begin{frame}{Springin modulit 2/2}
\begin{itemize}
\item Spring Dynamic Modules (OSGi)
\item Spring LDAP (template)
\item Spring Rich Client (Spring to Swing)
\item Spring.NET (Core, ADO.NET, NHibernate, ASP.NET, MSMQ)
\item Spring-Flex 
\item Spring Roo
\item \href[colorlinks=true]{http://www.springsource.org/extensions/list}{Spring Extensions} (Community driven) 
\end{itemize}
\end{frame}



% \begin{frame}{}
% \begin{itemize}
% \item
% \end{itemize}
% \end{frame}


\end{document}