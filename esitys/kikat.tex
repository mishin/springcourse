\newcommand{\AUTHOR}{Timo Friman, Henri Meltaus}
\newcommand{\DATE}{15.-16.9.2011}

\documentclass[hyperref={pdfauthor=\AUTHOR},14pt]{beamer}

\usepackage[finnish]{babel}
\usepackage[utf8]{inputenc}
\usepackage[T1]{fontenc}
\usepackage{listings}

\usecolortheme{crane}
\usetheme{default}

\definecolor{mygreen}{rgb}{0,0.4,0}
\definecolor{myid}{rgb}{0.1,0.1,0.1}
\lstdefinestyle{Java}{language=java,
 basicstyle=\small,%\ttfamily,
 numbers=left,stepnumber=1,numberstyle=\small\ttfamily,
 numbersep=5pt,frame=tlbr,extendedchars=true,
 commentstyle=\color{mygreen}\ttfamily,
 %% stringstyle=\color{red}\ttfamily,
 stringstyle=\color{magenta},
 keywordstyle=\color{violet}\bfseries,
 ndkeywordstyle=\color{yellow}\bfseries,
 identifierstyle=\color{myid},
 % sensitive=false,
 basicstyle=\scriptsize,
}

\author{\AUTHOR}
\newcommand{\TITLE}{Spring Framework \\ Kikkoja}
\title[\TITLE]{\TITLE}

\date{\DATE}

\begin{document}
\begin{frame}[plain]
\titlepage
\end{frame}

\begin{frame}[t,fragile]{MVC: WebContextAware}

Miten saat webappin reaalipolun selville?

\lstset{language=Java,style=Java}
\begin{lstlisting}
public class Example implements ServletContextAware {

@Override
public void setServletContext(ServletContext servletContext)
{
 String path = servletContext.getRealPath("resources/foo");
}

}
\end{lstlisting}
\end{frame}

\begin{frame}[t,fragile]{Tehtaita, tehtaita}
\begin{itemize}
\item Springin voi nähdä itsessään tehtaana.
\item Factory bean, spring bean factory, file autowire problem
\item esimerkkinä fopfileconfig
\end{itemize}
\end{frame}

\begin{frame}{Parhaat käytännöt}
\begin{itemize}
\item \url{http://springtips.blogspot.com/2007/06/twelve-best-practices-for-spring-xml.html}
\end{itemize}
\end{frame}

\begin{frame}{Lähteitä}
\begin{itemize}
\item Springin oma
  \href[colorlinks=true]{http://forum.springsource.org/}{forum} ja
  \href[colorlinks=true]{http://static.springsource.org/spring/docs/3.0.x/spring-framework-reference/html/index.html}{dokumentaatio}
\item Kirjoja
\begin{itemize}
\item Spring in Action
\item AspectJ in Action
\end{itemize}
\end{itemize}
\end{frame}

\end{document}