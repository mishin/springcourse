\newcommand{\AUTHOR}{Timo Friman, Henri Meltaus}
\newcommand{\DATE}{15.-16.9.2011}

\documentclass[hyperref={pdfauthor=\AUTHOR},14pt]{beamer}

\usepackage[finnish]{babel}
\usepackage[utf8]{inputenc}
\usepackage[T1]{fontenc}
\PassOptionsToPackage{hyphens}{url}\usepackage{listings}
\usepackage[hyphens]{url}

\usecolortheme{crane}
\usetheme{default}

\definecolor{mygreen}{rgb}{0,0.4,0}
\definecolor{myid}{rgb}{0.1,0.1,0.1}
\lstdefinestyle{Java}{language=java,
 basicstyle=\small,%\ttfamily,
% numbers=left,stepnumber=1,numberstyle=\small\ttfamily,
 numbersep=5pt,frame=tlbr,extendedchars=true,
 commentstyle=\color{mygreen}\ttfamily,
 %% stringstyle=\color{red}\ttfamily,
 stringstyle=\color{magenta},
 keywordstyle=\color{violet}\bfseries,
 ndkeywordstyle=\color{yellow}\bfseries,
 identifierstyle=\color{myid},
 % sensitive=false,
 basicstyle=\scriptsize,
}

\lstset{ % set listing language
language=XML,
breaklines=true,
breakautoindent=false,
breakatwhitespace=true,
frame=single,
%basicstyle=\tiny,
basicstyle=\scriptsize,
showspaces=false,               % show spaces adding particular underscores
showstringspaces=false,
resetmargins=true
}

\author{\AUTHOR}
\newcommand{\TITLE}{Spring Framework: Tyyppimuunnokset}
\title[\TITLE]{\TITLE}

\date{\DATE}

\definecolor{links}{HTML}{2A1B81}
\hypersetup{colorlinks,linkcolor=,urlcolor=links}

\begin{document}
\begin{frame}[plain]
\titlepage
\end{frame}

\begin{frame}[t,fragile]{JavaBeans}
\begin{itemize}
\item Vanha standardi, mutta saattaa silti yllättää housut kintuissa

\begin{lstlisting}[style=Java]
public void setFoo(Bar bar);
public void setFoo(InputStream is);
public Bar getFoo();
\end{lstlisting}

\item Spring toimii JavaBeans-standardin pohjalta:

\item Springin XML-konfiguraatiossa kaikki Stringeinä $\rightarrow$
  Tyyppimuunnos tarvitaan, mahdollisesti myös tieto, mitä tyyppiä kohdeobjekti on
\end{itemize}
\end{frame}

\begin{frame}{PropertyEditorit 1/2}
\begin{itemize}
\item \href[colorlinks=true]{http://download.oracle.com/javase/6/docs/api/java/beans/PropertyEditor.html?is-external=true}{java.beans.PropertyEditor} huolehtii Objectin esittämisestä
  Stringinä. Esimerkkinä Date, jonka toString() ei ole välttämättä se,
  jota halutaan sovelluksessa näyttää.
\item Spring käyttää sisäisesti PropertyEditoreja asettaessaan
  beaneihin arvoja ja samoin MVC käyttää sitä HTTP-pyyntöjen
  parametrien kanssa.
\item Springissä on automaattisesti tietyt perusmuunnokset käytössä: 
\href[colorlinks=true]{http://static.springsource.org/spring/docs/3.0.x/javadoc-api/org/springframework/beans/propertyeditors/package-summary.html}{org.springframework.beans.propertyeditors}
\end{itemize}
\end{frame}

\begin{frame}{PropertyEditorit 2/2}
\begin{itemize}
\item com.bar.Foo & com.bar.FooEditor $\rightarrow$ automaattinen rekisteröinti
  JavaBeans-infran kautta
\item Tarvittaessa omia PropertyEditoreita voi rekisteröidä vaikkapa
  CustomEditorConfigurerilla tai PropertyEditorRegistrarilla
\item Erilaiset ApplicationContextit rekisteröivät erilaisia omia
  PropertyEditoreita automaattisesti hoitaakseen resurssien etsimisen
  ja lataamisen
\end{itemize}
\end{frame}

\begin{frame}{ConversionService}
\begin{itemize}
\item \href[colorlinks=true]{http://static.springsource.org/spring/docs/3.0.x/javadoc-api/org/springframework/core/convert/ConversionService.html}{core.convert.ConversionService}
\item vaihtoehto PropertyEditoreille, tuli uutena versiossa 3.0
\item Vahvasti tyypitetty Genericseillä
\item Converter SPI (Service Provider Interface, tuttu JDBC:stä)
\end{itemize}
\end{frame}

\begin{frame}{Formatter SPI}
\begin{itemize}

\item Tämäkin on uutuus Spring 3:ssa
\item Tarkoitettu lähinnä asiakaspuolen formatointiin, kun taas
  Converter on yleismukautin $\rightarrow$ käytetään esim. MVC:ssä
\item
  \href[colorlinks=true]{http://static.springsource.org/spring/docs/3.0.x/javadoc-api/org/springframework/format/package-summary.html}{org.springframework.format},
  alipaketit sisältävät valmiita Formattereita, esim. joda.
\item tarjoaa mahdollisuuden antaa formatterpatternit annotaatioina
\end{itemize}
\end{frame}

\begin{frame}[t, fragile]{Esimerkki konfiguraatiosta}
MVC-konfiguraatio

\begin{lstlisting}
<mvc:annotation-driven conversion-service="conversionService"/>
<bean id="conversionService"
class="org.springframework.format.support.
FormattingConversionServiceFactoryBean"/>
\end{lstlisting}
\end{frame}

\begin{frame}{Validoinnista pari sanaa...}
\begin{itemize}
\item Spring 3 tukee JSR-303 Bean Validation APIa kokonaan,
  javax.validation, Hibernate validator on RI.
\item Annotaatiopohjainen validointi
\item MVC:ssä käytössä
\end{itemize}
\end{frame}

\end{document}