
\newcommand{\AUTHOR}{Timo Friman, Henri Meltaus}
\newcommand{\DATE}{15.-16.9.2011}

\documentclass[hyperref={pdfauthor=\AUTHOR},14pt]{beamer}

\usepackage[finnish]{babel}
\usepackage[utf8]{inputenc}
\usepackage[T1]{fontenc}
\PassOptionsToPackage{hyphens}{url}\usepackage{listings}
\usepackage[hyphens]{url}

\usecolortheme{crane}
\usetheme{default}

\definecolor{mygreen}{rgb}{0,0.4,0}
\definecolor{myid}{rgb}{0.1,0.1,0.1}
\lstdefinestyle{Java}{language=java,
 basicstyle=\small,%\ttfamily,
% numbers=left,stepnumber=1,numberstyle=\small\ttfamily,
% numbersep=5pt,frame=tlbr,extendedchars=true,
 commentstyle=\color{mygreen}\ttfamily,
 %% stringstyle=\color{red}\ttfamily,
 stringstyle=\color{magenta},
 keywordstyle=\color{violet}\bfseries,
 ndkeywordstyle=\color{yellow}\bfseries,
 identifierstyle=\color{myid},
 % sensitive=false,
 basicstyle=\scriptsize,
}

\author{\AUTHOR}
\newcommand{\TITLE}{Spring Framework - kokoonpano}
\title[\TITLE]{\TITLE}

\lstset{ % set listing language
language=XML,
breaklines=true,
breakautoindent=false,
breakatwhitespace=true,
frame=single,
%basicstyle=\tiny,
basicstyle=\scriptsize,
showspaces=false,               % show spaces adding particular underscores
showstringspaces=false,
resetmargins=true
}

\date{\DATE}

\begin{document}
\begin{frame}[plain]
\titlepage
\end{frame}

\begin{frame}[t, fragile]{KONFIGUROINTI}
\begin{itemize}
\item Kolme tapaa: XML, Annotaatiot, Java Config
\item Eri tapoja voi yhdistellä
\end{itemize}
\end{frame}

\begin{frame}[t, fragile]{ NIMEÄMINEN}
\begin{itemize}
\item Yksilöivä nimi id- tai name-attribuutilla
\item Käytetään kun halutaan viitata beaniin konfiguraatiossa
\item Id-attribuutti
\begin{itemize}
\item Schema-validointi duplikaattien varalta
\item Ei erikoismerkkejä
\end{itemize}
\item Name-attribuutti
\begin{itemize}
\item Voi käyttää erikoismerkkejä
\item Aliaksien antaminen 
\end{itemize}
\item Spring generoi nimen, jos ei annettu
\end{itemize}
\begin{lstlisting}
<bean id="bean1" class="com.houston.HelloBean" />
<bean name="beanA, beanB" class="com.houston.WorldBean" />
\end{lstlisting}
\end{frame}

\begin{frame}[t, fragile]{INSTANTIOINTI}
\begin{itemize}
\item Oletus-konstruktorilla
\begin{lstlisting}
<bean class="com.houston.HelloBean" />
\end{lstlisting}
 \item Parametrisoidulla konstruktorilla
\begin{lstlisting}
<bean class="com.houston.HelloBean">
 <constructor-arg index="0" value="42" />
</bean>
\end{lstlisting}
 \item Staattisella factory-metodilla 
\begin{lstlisting}
<bean id="clientService" class="examples.ClientService" factory-method="createInstance" />
\end{lstlisting}
 \item Bean-instanssin factory-metodilla
\begin{lstlisting}
<bean id="serviceLocator" class="examples.DefaultServiceLocator" />
<bean id="clientService" factory-bean="serviceLocator" factory-method="createClientServiceInstance" />
\end{lstlisting}
\end{itemize}
\end{frame}
\begin{frame}[t, fragile]{PROPERTYT JA RIIPPUVUUDET }
\begin{itemize}
 \item Propertyjen asetus setter-metodeilla
 \item Automaattiset tyyppikonversiot
 \item Tuettu mm. listat, mapit
 \item Viittaukset beaneihin ref-attribuutilla
\begin{lstlisting}
<bean id="dog" class="com.houston.Dog" />
<bean class="com.houston.Person">
  <property name="age" value="4" />
  <property name="pet" ref="dog" />
</bean>
\end{lstlisting}
\end{itemize}
\end{frame}

\begin{frame}[t, fragile]{AUTOWIRING}
\begin{itemize}
 \item Spring injektoi riippuvuudet automaattisesti
 \item Vähentää konfiguraation määrää
 \item Oletuksena pois päältä
 \item autowire-attribuutilla valitaan moodi:
   byType, byName, constructor, no
\begin{lstlisting}
<bean class="com.houston.HelloBean" autowire="byType"  />
\end{lstlisting}
 \item autowire-candidate:lla voi estää beanin käytön autowiressä
\begin{lstlisting}
<bean class="com.houston.HelloBean" autowire-candidate="false"  />
\end{lstlisting}
\end{itemize}
\end{frame}
\begin{frame}[t, fragile]{SCOPET}
\begin{itemize}
 \item Singleton (oletus)
 -- Vain yksi bean
 \item Prototype
 -- Uusi bean aina tarvittaessa
 \item Request (vain web)
 \item Session (vain web)
 \item Global session (vain portlet)
 \item Thread scope
 -- Oletuksena ei käytössä
\begin{lstlisting}
<bean class="com.houston.HelloBean" scope="session"  />
\end{lstlisting}
\end{itemize}
\end{frame}
\begin{frame}[t, fragile]{RIIPPUVUUDET SCOPEJEN VÄLILLÄ}
\begin{itemize}
 \item Lyhytikäisempi käyttää pitkäikäisempää
 -- request -> session -> prototype -> singleton
 \item Mahdollista myös toisinpäin proxy-luokan avulla\\
 -- CGLIB \\
 -- JDK rajapinta-proxy     
\end{itemize}
\begin{lstlisting}
<aop:scoped-proxy/> <!--CGLIB-->
<aop:scoped-proxy proxy-target-class="false" /><!--JDK-->

<bean id="userPreferences" class="com.foo.UserPreferences" scope="session">
  <aop:scoped-proxy/>
</bean>
<bean id="userService" class="com.foo.SimpleUserService" scope="singleton">
  <property name="userPreferences" ref="userPreferences"/>
</bean>
\end{lstlisting}

\end{frame}

\begin{frame}[t, fragile]{ELINKAARI}
\begin{itemize}

 \item Call-back -metodit, joita kutsutaan eri vaiheissa
 \item Toteuttamalla rajapinta, esim.
\begin{itemize}
\item InitializingBean
\item DisposableBean
\item Lifecycle
\end{itemize}
 \item XML:ssä
\begin{lstlisting}
<bean class="examples.ExampleBeanA" init-method="init"/>
<bean class="examples.ExampleBeanB" destroy-method="cleanup"/>
\end{lstlisting}
\end{itemize}
\end{frame}
\begin{frame}[t, fragile]{PERIMINEN}
\begin{itemize}
 \item Konfiguraation uudelleen käyttäminen:\\
 -- abstract-attribuutti perittävään\\
 -- parent-attributti perivään   \\
\begin{lstlisting}
    <bean id="inheritedTestBean" abstract="true" class="org.springframework.beans.TestBean">
      <property name="name" value="parent"/>
      <property name="age" value="1"/>
    </bean>

    <bean id="inheritsWithDifferentClass" class="org.springframework.beans.DerivedTestBean"
        parent="inheritedTestBean">
      <property name="name" value="override"/>
    </bean>
\end{lstlisting}
\end{itemize}
\end{frame}
\begin{frame}[t, fragile]{KONFIGURAATION KUSTOMOINTI}
\begin{itemize}
 \item Konfiguraation kustomointi ennen beanien instantiointia
 \item Toteuta rajapinta org.springframework.beans.factory.config.BeanFactoryPostProcessor\\
 -- Spring tunnistaa rajapinnan toteuttavat beanit\\
 -- Suoritusjärjestykseen voi vaikuttaa toteuttamalla Ordered-rajapinta \\
 \item esim. Springin property-placeholdereiden korvaus xml-konfiguraatiossa 
\end{itemize}
\end{frame}
\begin{frame}[t, fragile]{BEANIEN KUSTOMOINTI}
\begin{itemize}
 \item Bean instanssien kustomointi ennen ja/tai jälkeen elinkaari-callbackien\\
 -- Beanin instantiointi, riippuvuudet yms.
 \item Toteuta rajapinta BeanPostProcessor \\
-- Spring tunnistaa rajapinnan toteuttavat beanit\\
 -- Suoritusjärjestykseen voi vaikuttaa toteuttamalla Ordered-rajapinta\\
 \item Esim. Springin AOP ja @Required-annotaatioiden tarkistus 
 \item AOP ei mahdollista BeanPostProcessor-luokissa
\end{itemize}
\end{frame}
\begin{frame}[t, fragile]{BEANIEN KUSTOMOITU INSTANTIOINTI}
\begin{itemize}
 \item Käytä jos beanin instantiointi on monimutkainen
 \item Toteuta rajapinta org.springframework.beans.factory.FactoryBean\\
 -- Spring tunnistaa rajapinnan toteuttavat beanit
\end{itemize}
\end{frame}
\begin{frame}[t, fragile]{KONTEKSIN PALJASTAMINEN}
\begin{itemize}
\item Normaalisti beanit eivät tiedä Spring-kontekstista mitään
 \item Toteuttamalla rajapinnan bean voi saada tietoonsa jotain kontekstista:\\
 -- ApplicationContextAware \\
 -- ApplicationEventPublisherAware \\
 -- BeanNameAware \\
 -- ResourceLoaderAware \\
 -- ServletConfigAware \\
 -- ServletContextAware \\
\end{itemize}
\end{frame}
\begin{frame}[t, fragile]{EXPRESSION LANGUAGE (SpEL) - API}
\begin{itemize}
 \item Objektigraafien manipulointiin ja kyselyihin
 \item Unified EL:n (esim. JSP) kaltainen
 \item Metodien kutsuminen, muuttujat, laskutoimitukset
 \item Voi käyttää XML:ssä ja annotaatioissa
 \item Toimii myös itsenäisenä API:na, ei vaadi muita Spring-kirjastoja  
\begin{lstlisting}[style=Java]
ExpressionParser parser = new SpelExpressionParser();
Expression exp = parser.parseExpression("'Hello World'");
String message = (String) exp.getValue();
\end{lstlisting}
\end{itemize}
\end{frame}
\begin{frame}[t, fragile]{EXPRESSION LANGUAGE (SpEL) - KÄYTTÖ }
\begin{itemize}
 \item Expression muodossa \#{ <expression string> }
 \item Voi sisältää mm. viittauksia beaneihin, propertyihin 
\begin{lstlisting}
<bean id="numberGuess" class="org.spring.samples.NumberGuess">
  <property name="randomNumber" value="#{T(java.lang.Math).random() * 100.0}"/>
</bean>

<bean id="shapeGuess" class="org.spring.samples.ShapeGuess">
  <property name="initialShapeSeed" value="#{numberGuess.randomNumber}"/>
</bean>
\end{lstlisting}
\end{itemize}
\end{frame}
\begin{frame}[t, fragile]{KONFIGUROINTI ANNOTAATIOILLA}
\begin{itemize}
 \item Beanien skannaus ja instantiointi classpathista XML:ssä
\begin{lstlisting}
<context:annotation-config />
<context:component-scan base-package="com.houston"/>
\end{lstlisting}
\end{itemize}
\end{frame}
\begin{frame}[t, fragile]{ANNOTAATIOT (Bean-määritykset)}
\begin{itemize}
 \item Merkkaa bean-luokat stereotype-annotaatioilla:\\
 @Component, @Repository, @Service, @Controller\\
 -- Beanin nimen voi myös antaa\\
 \item @Scope
 -- Määrittää beanin scopen (esim. singleton, prototype)
 \item @Qualifier
 -- Beanin qualifier
\end{itemize}
\end{frame}
\begin{frame}[t, fragile]{ANNOTAATIOT (riippuvuudet)}
\begin{itemize}
 \item @Required
\begin{itemize}
 \item Pakollinen property
 \item Setter, constructor
\end{itemize}
 \item @Autowired 
\begin{itemize}
 \item Autowire-riippuvuus
 \item  Required-attribuutilla voi määrittää onko riippuvuus pakollinen (true oletuksena)
 \item  Field, setter, constructor
 \item  Myös muihin, ei-setter -tyyppisiin, metodeihin
 \item  Suosi @Required:n sijasta
\end{itemize}
\end{itemize}
\end{frame}
\begin{frame}[t, fragile]{ANNOTAATIOT (riippuvuudet) }
\begin{itemize}
 \item @Qualifier
\begin{itemize}
 \item  Käytä tarkentamaan @Autowired-määritystä, jos siihen on useampi ehdokas
 \item XML-konfiguraatiossa beanin qualifierin määrittää beanin sisään tuleva <qualifier>
 \item  Oletuksena beanin id on myös sen qualifier, jos sitä ei ole erikseen määritetty
\end{itemize}
\end{itemize}
\end{frame}
\begin{frame}[t, fragile]{ANNOTAATIOT (elinkaari)}
\begin{itemize}
 \item @PostConstruct
 -- Metodi, jota kutsutaan kun bean on instantioitu
 \item @PreDestroy
 -- Metodi, jota kutsutaan juuri ennen beanin poistoa
\end{itemize}
\end{frame}
\begin{frame}[t, fragile]{ANNOTAATIOT (muut)}
\begin{itemize}
\item @Value
\begin{itemize}
\item Esim. propertyissä määritetyt muuttujat
\item Hyödyntää Spring EL:ää
\end{itemize}
\end{itemize}
\end{frame}
\begin{frame}[t, fragile]{KONFIGUROINTI OHJELMALLISESTI}
\begin{itemize}
 \item Java Configin luonti stand-alone -sovelluksessa\\
 -- AnnotationConfigApplicationContext
 \item Sama web-sovelluksessa\\
 -- AnnotationConfigWebApplicationContext\\
 -- Rekisteröinti web.xml:ssä\\
 \item Konfiguraatioluokkien tuominen olemassa olevaan kontekstiin\\
 -- Luokkien merkkaus @Configuration-annotaatiolla\\
 -- Instantiointi <context:component-scan /> xml-elementin avulla\\
\end{itemize}
\end{frame}
\begin{frame}[t, fragile]{BEANIEN INSTANTIOINTI KONFIGURAATIOLUOKASTA}
\begin{itemize}
\item Yksi metodi per instantioitava bean
\begin{itemize}
\item Merkattu @Bean-annotaatiolla
\item attribuutit: initMethod, destroyMethod, name
\item Beanin @Scope-annotaatiolla
\end{itemize}
\end{itemize}
\end{frame}
\begin{frame}[t, fragile]{KONFIGURAATIOIDEN IMPORTTAUS}
\begin{itemize}
 \item @Import
 -- Importoi konteksiin toisen java-konfiguraation
 \item @ImportResource
 -- Importoi kontekstiin XML-konfiguraation
\end{itemize}
\end{frame}

\begin{frame}[t, fragile]{KONFIGURAATIOTAPOJEN EROT}
XML\\
\scriptsize
 -- Samasta luokasta useampi instanssi eri konfiguraatioilla\\
 -- Muidenkin kuin omien luokkien konfigurointi\\
 -- Voidaan ylikirjoittaa annotaatiot\\
 -- Propertyjen asettaminen\\

\large Annotaatiot\\
\scriptsize
 -- Sama konfiguraatio jokaiselle luokan instanssille\\
 -- Konfiguraatio kiinteästi osana beania\\
 -- Rajoittuneempi\\

\normalsize Java Config\\
\scriptsize
 -- Ei rajoituksia, kaiken voi tehdä\\
 -- Työläämpi?\\
\end{frame}


\end{document}