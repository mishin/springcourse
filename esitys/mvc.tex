\newcommand{\AUTHOR}{Timo Friman, Henri Meltaus}
\newcommand{\DATE}{15.-16.9.2011}

\documentclass[hyperref={pdfauthor=\AUTHOR},14pt]{beamer}

\usepackage[finnish]{babel}
\usepackage[utf8]{inputenc}
\usepackage[T1]{fontenc}
\PassOptionsToPackage{hyphens}{url}\usepackage{listings}
\usepackage[hyphens]{url}

\usecolortheme{crane}
\usetheme{default}

\definecolor{mygreen}{rgb}{0,0.4,0}
\definecolor{myid}{rgb}{0.1,0.1,0.1}
\lstdefinestyle{Java}{language=java,
 basicstyle=\small,%\ttfamily,
 numbers=left,stepnumber=1,numberstyle=\small\ttfamily,
 numbersep=5pt,frame=tlbr,extendedchars=true,
 commentstyle=\color{mygreen}\ttfamily,
 %% stringstyle=\color{red}\ttfamily,
 stringstyle=\color{magenta},
 keywordstyle=\color{violet}\bfseries,
 ndkeywordstyle=\color{yellow}\bfseries,
 identifierstyle=\color{myid},
 % sensitive=false,
 basicstyle=\scriptsize,
}

\lstset{ % set listing language
language=XML,
breaklines=true,
breakautoindent=false,
breakatwhitespace=true,
frame=single,
%basicstyle=\tiny,
basicstyle=\scriptsize,
showspaces=false,               % show spaces adding particular underscores
showstringspaces=false,
resetmargins=true
}

\definecolor{links}{HTML}{2A1B81}
\hypersetup{colorlinks,linkcolor=,urlcolor=links}

\author{\AUTHOR}
\newcommand{\TITLE}{Spring Framework MVC}
\title[\TITLE]{\TITLE}

\date{\DATE}

\begin{document}
\begin{frame}[plain]
\titlepage
\end{frame}

\begin{frame}[t, fragile]{Perusteet}
\begin{itemize}
\item DispatcherServlet delegoi sivupyynnöt controller-luokille
\item Sivupyynnöt linkitetään controllerien metodeihin
\item Controller muodostaa näkymän tai kirjoittaa vastauksen suoraan responseen
\end{itemize}
\end{frame}

\begin{frame}[t, fragile]{Konfigurointi}
\begin{itemize}
\item DispatcherServlet web.xml:ään
\item Spring-konteksti oletuksena WEB-INF/[servlet name]-servlet.xml
\item <mvc:annotationDriven> hyvä oletuskonfiguraatio
-- Konversiot, validointi
\item <mvc:resources> staattiset resurssit
\end{itemize}
\end{frame}

\begin{frame}[t, fragile]{Controllerit}
\begin{itemize}
\item Tavallisia Spring-beaneja
\item Merkattu @Controller-annotaatiolla
\item Sivupyynnöt käsitellään @RequestMapping-annotaatiolla merkityissä metodeissa
\item Metodin paluuarvon perusteella muodostetaan vastaus selaimelle
\end{itemize}
\end{frame}
\begin{frame}[t, fragile]{Sivupyyntöjen linkitys controllerien metodeille}
\begin{itemize}
\item@RequestMapping-annotaatiolla luokka- ja metodi-tasolla
\item Useita sääntöjä:
\begin{itemize}
\item Url (esim "/movies/*")
\item Request-parametrit
\item Headerit
\item Http-metodit (GET, POST...)
\end{itemize}
\end{itemize}
\end{frame}

\begin{frame}[t, fragile]{Controller-metodien argumentit}
\begin{itemize}
\item Request-parametrit
\item REST-tyyliset urlin osat, esim. "/users/{userId}"
\item Headerit
\item Cookiet
\item HttpServletRequest
\item HttpServletResponse
\item HttpSession
\item Custom argumentit WebArgumentResolver-rajapinnan avulla
\item Argumenttien automaattinen konvertointi
\end{itemize}
\end{frame}

\begin{frame}[t, fragile]{Selaimelle palautettava vastaus}
\begin{itemize}
\item Controller palauttaa näkymän nimen, joka linkitetään näkymään
-- JSP, PDF...
\item Controller palauttaa POJOn, joka sarjallistetaan
-- JSON, XML...
\item Controller kirjoittaa vastauksen suoraan responseen
-- Tiedostot kuten kuvat...
\end{itemize}
\end{frame}

\begin{frame}[t, fragile]{Näkymät}
\begin{itemize}
\item ModelAndView sisältää modelin ja näkymän nimen
\item Useita ViewResolver-toteutuksia
\item Model sisältää muuttujat, jotka käytössä näkymässä
\item Käytä selaimessa esitettävien sivujen näyttämiseen
\end{itemize}
\end{frame}

\begin{frame}[t, fragile]{Sarjallistettavat POJOt}
\begin{itemize}
\item Metodin paluutyyppi merkataan @ResponseBody-annotaatiolla
\item Sarjallistaminen HttpMessageConverter-luokilla
-- Selaimen pyytämän content typen mukaan
\item Käytä Ajaxin ja WS/REST -clientien kanssa 
\end{itemize}
\end{frame}

\begin{frame}[t, fragile]{Lomakkeet 1/2}
\begin{itemize}
\item Lomakkeen kentät backing objectiin
\item Validointi @Valid-annotaatiolla
\item @InitBinder-annotaatiolla merkatussa metodissa
-- Omien konvertterit ja validaattorien rekisteröinti
\item Springin omat jstl-tagit
\end{itemize}
\end{frame}

\begin{frame}[t, fragile]{Lomakkeet 2/2}
\begin{itemize}
\item BindingResult sisältää validointi- ja konversiovirheet
\item Jos virheitä
-- Näytä lomakesivu uudestaan virheviestien kanssa
\item Ei virheitä 
-- Redirect seuraavaan näkymään (tuplapostauksen estäminen)
\end{itemize}
\end{frame}

\begin{frame}[t, fragile]{Virheiden käsittely}
\begin{itemize}
\item Controllerissa
\begin{itemize}
\item Merkitse virheet käsittelevä metodi @ExceptionHandler-annotaatiolla
\item Käsittely poikkeuksen tyypin mukaan
\end{itemize}
\item Sovelluksen tasolla
\begin{itemize}
\item DefaultHandlerExceptionResolver 
\item oma toteutus
\end{itemize}
\end{itemize}
\end{frame}

\begin{frame}[t, fragile]{Interceptorit}
\begin{itemize}
\item Ennen ja jälkeen controllerin metodin sekä koko sivupyynnön lopuksi
\item Linkitys urlien mukaan
\item Rekisteröinti esim. <mvc:interceptors /> elementillä
\end{itemize}
\end{frame}

\begin{frame}[t, fragile]{Testaus}
\begin{itemize}
\item Helppo yksikkötestata, koska controllerit ovat POJOja
\item Valmis setti mock-luokkia
\item Integraatiotestaus
\begin{itemize}
\item Selenium
\item HtmlUnit
\item JWebUnit
\end{itemize}
\end{itemize}
\end{frame}

\end{document}